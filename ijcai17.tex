%%%% ijcai17.tex

\typeout{IJCAI-17 Instructions for Authors}

% These are the instructions for authors for IJCAI-17.
% They are the same as the ones for IJCAI-11 with superficical wording
%   changes only.

\documentclass{article}
% The file ijcai17.sty is the style file for IJCAI-17 (same as ijcai07.sty).
\usepackage{ijcai17}

% Use the postscript times font!
\usepackage{times}

\usepackage{amssymb}
\usepackage{tikz}
%\usetikzlibrary{bayesnet}
\usepackage{diagbox}
\usepackage{latexsym}
\usepackage{graphicx}
\usepackage{graphics}
\usepackage{amsmath}
\usepackage{natbib}
\usepackage{amsthm}       %% The amsthm package provides extended theorem environments
\usepackage{multirow}
\usepackage{algorithmic}
\newcommand{\dir}{\text{Dir}}
\newcommand{\mult}{\text{Multi}}
\newtheorem{example}{Example}
\usepackage{amstext}
\usepackage{hyperref}
\newtheorem{definition}{Definition}
%\newcommand{\upcite}[1]{\textsuperscript{\textsuperscript{\cite{#1}}}}
%\usepackage{algorithm}
%\usepackage{algorithmic}
\makeatletter
\newif\if@restonecol
\makeatother
\let\algorithm\relax
\let\endalgorithm\relax
\usepackage[linesnumbered,ruled,vlined]{algorithm2e}
\usepackage{enumerate}

% the following package is optional:
%\usepackage{latexsym}

% Following comment is from ijcai97-submit.tex:
% The preparation of these files was supported by Schlumberger Palo Alto
% Research, AT\&T Bell Laboratories, and Morgan Kaufmann Publishers.
% Shirley Jowell, of Morgan Kaufmann Publishers, and Peter F.
% Patel-Schneider, of AT\&T Bell Laboratories collaborated on their
% preparation.

% These instructions can be modified and used in other conferences as long
% as credit to the authors and supporting agencies is retained, this notice
% is not changed, and further modification or reuse is not restricted.
% Neither Shirley Jowell nor Peter F. Patel-Schneider can be listed as
% contacts for providing assistance without their prior permission.

% To use for other conferences, change references to files and the
% conference appropriate and use other authors, contacts, publishers, and
% organizations.
% Also change the deadline and address for returning papers and the length and
% page charge instructions.
% Put where the files are available in the appropriate places.

\title{Knowledge Graph Embedding With Attentional Triple Context}
\author{Huan Gao$^{1,2}$, Jun Shi$^{1,2}$, Guilin Qi$^{3}$\\
$^{1}$School of Computer Science and Engineering,
Southeast University, Nanjing, China  \\
%qiuji@njupt.edu.cn\\
}

\begin{document}

\maketitle

\begin{abstract}
    Knowledge graph embedding can represents entities and relations with efficient low-dimensional embedding vectors. The outstanding performance on knowledge graph completion has led to an increase in the knowledge graph embedding research. State-of-the-art knowledge graph embedding approaches treat each triple independent and neglect structure information. However, as a fact, the rich graph features in knowledge graph can be considered as contexts of a triple which contain large information to describe entities and relations. In this paper, we proposes an Attentional-Triple-Context-based knowledge Embedding model(ATCE), which formulates a local structures around a triple as a context. For each triple, two kinds of structure information are considered as its context, which we refer to \emph{as triple context} : 1) Neighbor context is the outgoing relations and neighboring entities of an entity; 2) Path context is connective relation paths between a pair of entities, both of which contains rich useful and unrelated information for entities and relations. ATCE learns embedding for entities and relations with a attention mechanism and is expected to select the useful information in triple context. The experimental results show that our model outperforms the state-of-the-art methods for link prediction and entity prediction.
\end{abstract}



\section{Introduction}
Recent advances in information extraction have led to huge Knowledge graphs(KGs), such as DBpedia~\cite{DBpedia2007}, YAGO~\cite{Yago2007} and NELL~\cite{NELL2015}. These KGs contain facts which represent relations between entities as triples $<h,r,t>$. A triple indicates that entities $h$ and $t$ are connected by relation $r$. Even a KG contains a very large number of triples, it is still far from complete. The completeness of KGs damages their usefulness in downstream real applications. Knowledge graph completion or link predictions is thus important approaches for populating existing KGs.

Knowledge graph embedding models for KG completion have attracted much attention, due to their outstanding performance on various applications that require machines to recognize and understand queries and their semantics. These embedding models are to represent entitles and relations in a KG into a low dimensional continuous vector space, where these vectors contain rich semantic information, and can benefit many downstream tasks especially knowledge graph completion or linked predictions.

Despite the success of previous approaches in KG embedding ~\cite{BordesUGWY13} ~\cite{WangZFC14}, most of the mainly models treat triples individually, ignoring lots of information implicitly provided by the structure of the KG. In fact, triples and the relations among them have abundant information that can be further used for inference. Recently, several authors have addressed this issue by incorporating relation path information into model learning ~\cite{LinLLSRL15} ~\cite{Toutanova16} learning and have shown that the relation paths between entities in KGs provide useful information and improve the task of KG completion. These approaches only consider relation information while missing more structure information which contain rich clues for inference. For instance if we know that Ben Affleck has won an Oscar award and Ben Affleck lives in Los Angeles, then this can help us to predict that Ben Affleck is an actor or a film maker, rather than a lecturer or a doctor.

In order to utilize the structure information, we present a novel approach to embed a knowledge graph named Attentional-Triple-Context-based Knowledge Embedding model(ATCE). ATCE utilizes and chooses the proper context of each triple in the knowledge graph. We define triple context consisting of neighbor context and path context, and define a new score function to evaluate the correlation between a triple and its contexts. Instead of using each triple independently, we incorporate triple context into the score function which is used to evaluate the confidence of a triple. 

The advantages of our approach are three-fold:
1) We embed a triple by utilizing a local subgraph around a triple instead of a set of independent triples, and extract two kinds of contexts named triple context.

2) Based on the triple context, we proposed a novel embedding learning approach which named ATCE and a new loss function which converts the score function in TransE to a conditional probability.

3) In order to overcome the noisy data in the triple context of a triple, an attention mechanism in our approach is proposed to choose the proper information for embedding. In the meanwhile, the attention mechanism can learn the representation power of different neighbor entities and connective from in its context.

Finally, we have conducted preliminary experiments on two benchmark data sets and assessed our method on link prediction task and triple classification. In the experiments we show chosen context through the attention mechanism to improve the effectiveness of this mechanism. The experimental results show impressive improvements on predictive accuracy compared to other baselines which include TransE~\cite{BordesUGWY13}, TransH~\cite{WangZFC14}, TransR~\cite{LinLSLZ15} , CTransR~\cite{LinLSLZ15}, PTransE~\cite{LinLLSRL15} and GAKE~\cite{FengHYZ16}. .


\section{Triple Context}\label{sec:pre}
%In this section, we introduce our model that learns embeddings of entities and relations with the help of triple context in the knowledge graph.
Firstly, we introduce some notations that are used in this paper. Let $\mathcal{K}$ be a knowledge graph, $\mathcal{E}$ and $\mathcal{R}$ the set of all entities and relations respectively in $\mathcal{K}$. Each triple is denoted as $(h, r, t)$, in which $h$ is the head entity, $t$ is the tail entity and $r$ is the relation between $h$ and $t$. The embeddings of each entity and relation are denoted in bold, e.g., $\bm{\mathrm{h}}$ is the embedding of $h$. All the embeddings are in $d$-dimensional space $\mathbb{R}^d$. Our goal is to learn embeddings of all entities and relations, which is denoted as $\Theta$. In the following subsections, we define neighbor context and path context, and then give the framework of our model.

\subsection{Neighbor Context}
Neighbor context of an entity is the surroundings of it in KG. It is the local structure that interacts most with the entity and can reflect various aspects of the entity. Specifically, given an entity $e$, the neighbor context of $e$ is a set $C_N(e)=\{(r,t)|\forall r, t, (e,r,t)\in\mathcal{K}\}$, where $r$ is an outgoing edge (relation) from $e$ and $t$ is the entity it reaches through $r$. In other words, the neighbor context of $e$ is all the \textit{relation-tail} pairs appearing in triples with $e$ as the head. For example, as shown in Figure~\ref{pic1}, the neighbor context of entity $h$ is $C_N(h)=\{(r_4, e_1), (r_3, e_2), (r_2, e_3), (r_1, e_8), (r_1, e_{10})\}$. We predict the appearance of an entity based on its neighbor context in our model, as a measurement of the compatibility of the entity and its neighbor context.

\begin{figure}
  \includegraphics[width=0.45\textwidth]{pic1.png}
  \caption{An illustration of the \emph{triple context} of a triple $(h,r,t)$ in a knowledge graph.}
  \label{pic1}
\end{figure}


\subsection{Path Context}
Path context of a pair of entities is the set of paths that starts from an entity to the other in a KG. It is helpful in modeling the relation and capturing interactions between the pair of entities. Given a pair of entities $(h,t)$, the path context of $(h,t)$ is a set $C_P(h,t)=\{p_i | \forall r_{m_1}, \cdots, r_{m_i}, e_1, \cdots, e_{m_i-1},$ $p_i=(r_{m_1}, \cdots, r_{m_i}), (h,r_{m_1},e_1)\in\mathcal{K}, \cdots, (e_{m_i-1}, r_{m_i}, t)\in\mathcal{K}\}$, where $p_i=$ is a list of relations (labeled edges) through which it can traverse from $h$ to $t$, $m_i$ is the length of path $p_i$. In Figure~\ref{pic1}, the path context between $h$ and $t$ is $C_P(h,t) = \{(r_1, r_2), (r_2, r_1, r_2)\}$. We use the path context to predict the tail entity of a triple given the head entity.

\section{Knowledge Graph Embedding With Attentional Triple Context} \label{sec:pro}
So far, we have introduced neighbor context and path context, based on which we can define triple context. The triple context of triple $(h,r,t)$ is composed of the neighbor context of the head entity $h$ , the path context of the entity pair $(h,t)$, which can be formalized as:
\begin{equation}\label{triple context}
  C(h,r,t) = C_N(h) \cup C_P(h, t)
\end{equation}

The triple context of a triple can be considered to embody the surrounding structures of it in the graph, which makes the model aware of the information contained in graph structures.

We then introduce our approach in detail. In general KG embedding models, the score function of a triple is only related to the embeddings of entities and relations. For example, TransE defines the score function as $f_{TransE}(h,r,t)=\|\bm{\mathrm{h}} + \bm{\mathrm{r}} -\bm{\mathrm{t}}\|_{L_1/L_2}$. In our method, triple context is introduced in the score function. Given a candidate triple $(h,r,t)$, the score function is the conditional probability that the triple holds given the triple context and all the embeddings, as follows:
\begin{equation}\label{score_function}
  f(h,r,t) = P((h,r,t)|C(h,r,t);\Theta)
\end{equation}
where $C(h,r,t)$ is the triple context of $(h,r,t)$. A higher score of a triple indicates that it holds to a greater extent.

We define an objective function by maximizing the joint probability of all triples in knowledge graph $\mathcal{K}$, which can be formulated as:
\begin{align} \label{joint_prob}
  P(\mathcal{K}|\Theta) &= \prod_{(h,r,t)\in \mathcal{K}} f(h,r,t)
\end{align}

For the score function in Eq.~\eqref{score_function}, we use conditional probability formula to decompose the probability $P((h,r,t)|C(h,r,t);\Theta)$ as:
\begin{align} \label{decomposition}
  \begin{split}
    f(h,r,t) &= P(h|C(h,r,t);\Theta) \\
    & \cdot P(t,r|C(h,r,t),h;\Theta)
  \end{split}
\end{align}
where the evaluation of the whole triple is decomposed into two parts. The probabilities that $h$, $t$ and $r$ appear given respective condition are determined in turn in these two parts.
% $h$ is determined first based on triple context, then $t$ is determined based on $h$ and triple context, and finally $r$ likewise.

The first part $P(h|C(h,r,t);\Theta)$ in Eq.~\eqref{decomposition} represents the conditional probability that $h$ is the head entity given the triple context and all the embeddings. Since whether $h$ appears is decided mostly by the neighboring structures of $h$ in the KG, we can approximate $P(h|C(h,r,t);\Theta)$ as $P(h|C_N(h);\Theta)$, where $C_N(h)$ is the \emph{neighbor context} of $h$ in the KG. The approximated probability $P(h|C_N(h);\Theta)$ can be considered as the compatibility between $h$ and its neighbor context, it is formalized as a softmax-like representation, which is also used in~\cite{DBLP:conf/emnlp/WangZFC14} and has been validated, as follows:
\begin{align} \label{P_h}
  P(h|C_N(h);\Theta) = \frac {\exp(g_N(h, C_N(h)))} {\sum_{h' \in \mathcal{E}} \exp(g_N(h', C_N(h)))}
\end{align}
where $g_P(\cdot, \cdot)$ is the function that describes the correlation between an arbitrary entity $h'$ and entity context of the specific entity $h$. In reality, different neighbors may have different power of influence to represent $h$. For example, a target entity is $Terminate2:JudgementDay$ which is a famous in 1991. The director of this entity  or the type of this entity may more important than others. In order to unrelated entities and obtain the correlation, we first obtain $a_i$ which means the represent power between the $ith$ neighbor entity in $C_N(h)$ and an arbitrary triple $<h',r,t>$. Inspired by score function of TransE, we substitute the neighbor entities in $C_N(h)$ for the head $h'$ in triple to compute $a_i$ and then we define $a_i$
\begin{align} \label{a_i}
a_i = \|t_{n_i} - r_{n_i} + r - t\|
\end{align}

Where $t_{n_i}$ is the $ith$ neighbor entity in $C_N(h)$ and $r_{n_i}$ is the relation between $t_{n_i}$ and $h'$.

Then We use attention model $\alpha_i$ to represent how $h'$ selectively focuses on $C_N(h)$, If the value of $\alpha_i$ is higher, the corresponding neighbor entity in $C_N(h)$ is more important.
\begin{equation}\label{alpha_i}
\alpha_i = \frac {\exp(-a_i)} {\sum_{j} \exp(-a_j)}
\end{equation}

In Eq.~\eqref{g_N}, given a neighbor context $C_N(h)$, the embedding vector of each neighbor has different weights by further considering attention mechanism. Finally, based on the attention results we obtain the correlation between an arbitrary entity $h'$ and neighbor context $C_N(h)$:
\begin{align} \label{g_N}
g_N(h��, C_N(h)) = -\sum_{i} \alpha_{i} \|\mathbf{h'} + \mathbf{r_{n_i}} - \mathbf{t_{n_i}}\|
\end{align}

The second part $P(r,t|C(h,r,t), h; \Theta)$ in Eq.~\eqref{decomposition} is the conditional probability that $t$ is the tail entity and $r$ is the relation given the head entity $h$, triple context and all the embeddings. In this part we introduce path context that means $t$ could be related to $h$ through a potential connective path in a knowledge graph. In the second part two kinds of relatedness should be considered that one is the relatedness between $h$ and $t$ in a a potential connective path $p_i$. And the other is the relatedness between $r$ and $p_i$.  Then We introduce path context among $h$ , $t$ and $r$ to measure the relatedness of them and approximate $P(r,t|C(h,r,t),h;\Theta)$ as $P(t|C_P(h,t),h;\Theta)$, where $C_P(h,t)$ is the \emph{path context} between $h$ and $t$. The approximated probability $P(r,t|C_P(h,t),$ $h;\Theta)$ is formalized as follows:

\begin{small}
\begin{equation}\label{P_t}
P(r,t|C_P(h, t), h;\Theta) \!=\! \frac {\exp(g_P(r,t, C_P(h, t)))} {\sum_{r' \in \mathcal{R}, t' \in \mathcal{E}} \exp(g_P(r', t', C_P(h, t)))}
\end{equation}
\end{small}

where $g_P(\cdot, \cdot)$ is a function of correlation among an arbitrary entity $t'$  an arbitrary relation $r$ and path context of the specific entity pair $(h,t)$. Similar to the neighbor context, given a triple $<h,r,t>$ different pathes in a path context $C_P(h, t)$ have different power of influence the triple. For example when predicting the entity $Englis$, relations like $locate\_in\_Country$ will have less attentions, and the relation $Speak\_Language$  will have greater attention to represent $Engilsh$. In order to obtain $g_P(\cdot, \cdot)$, we firstly calculate the correlation $b_i$ between path $p_i$ in $C_P(h, t)$ and the triple:
\begin{equation}\label{f2}
b_i = \|\mathbf{h} + \mathbf{p_i} - \mathbf{t}\|
\end{equation}

 Where $\bm{\mathrm{p_i}}$ composes all relations in $p_i$ into a single vector by summing over all their embeddings and this approach is also used in Ptrans\cite{}. For example, for path $p_i=(r_{m_1}, \cdots, r_{m_i})$, the embedding of it is $\bm{\mathrm{p}}_i = \bm{\mathrm{r}}_{m_1} + \cdots + \bm{\mathrm{r}}_{m_i}$. Eq.~\eqref{f2} has a similar meaning with Eq.~\eqref{a_i}.

We then choose important pathes from $C_P(h, t)$ through the attention mechanisms. For each $p_i$, the weight $\beta_{i}$ is then defined as
\begin{align}\label{beta_i}
\beta_{i} = \frac {\exp(-b_i)} {\sum_j \exp(-b_j)}
\end{align}
If $p_i$ is more closer to $h$ and $t$, $\beta_{i}$ will indicate $p_i$ have greater attentions on translating from $h$ to $t$.

Finally given an  arbitrary entity $t'$ and an arbitrary relation $r'$, the correlation $g_P(\cdot, \cdot)$ between $t'$,$r'$ and $C_P(h, t)$ is:
\begin{equation}\label{gp}
g_P(r', t', C_P(h,t)) = -\sum_{i} \beta_i  (\|\mathbf{h} + \mathbf{p}_i - \mathbf{t'} \| + \|\mathbf{p}_i - \mathbf{r'}\|)
\end{equation}

In Eq.~\eqref{gp}, given a path context $C_P(h,t)$, the embedding vector of each path $p_i$ has different weight $\beta_i$. Eq.~\eqref{gp} has two parts, the first part is  $\|\mathbf{h} + \mathbf{p}_i - \mathbf{t'} \|$ that indicates the correlation between $t'$ and $p_i$. Similarly, the second part is $|\mathbf{p}_i - \mathbf{r'}\|$ is the correlation between $r'$ and $p_i$. If $p_i$ is more related to a triple, the probability of Eq.~\eqref{P_t} will be greater.

To utilize these two kinds of context, we combine them by jointly maximizing the probability in Eq.~\eqref{decomposition} of a triple which is exist in a knowledge graph. $P(h|C(h,r,t);\Theta)$ and $P(t,r|C(h,r,t),h;\Theta)$ can be approximated as $P(h|C_N(h);\Theta)$ and $P(r,t|C_P(h,t),h;\Theta)$, respectively. Thus Eq.~\eqref{decomposition} Thus, Eq.~\eqref{decomposition} can be approximated as:
\begin{equation}\label{decomposition_approx}
  f(h,r,t) \approx P(h|C_N(h);\Theta) \cdot P(t,r|C_P(h,t),h;\Theta)
\end{equation}
in which way the neighbor context and the path context of a triple are incorporated.


\subsection{Model Learning}
By feasible approximation, the score function is transformed to Eq.~\eqref{decomposition_approx}, each part is represented in softmax form as Eq.~\eqref{P_h}, Eq.~\eqref{P_t} and Eq.~\eqref{P_r}. However, it is impractical to compute these softmax functions directly because of high computational overhead. Hence, we adopt negative sampling, which is proposed in \cite{DBLP:conf/nips/MikolovSCCD13} to approximate full softmax function efficiently, to approximate softmax functions in our model. Taking $P(h|C_N(h);\Theta)$ in Eq.\eqref{P_h} as an example, it is approximated via negative sampling as follows:
\begin{equation}\label{approximation}
\begin{split}
-\log P(\mathcal{K} | \Theta)
& = -\sum_{(h,r,t) \in \mathcal{K}} [\log \sigma(g_N(h, C_N(h))) \\
&+ \sum_{h'} \log \sigma(-g_N(h', C_N(h))) \\
&+ \log \sigma(g_P(r, t, C_P(h, t))) \\
&+ \sum_{r', t'} \sigma(- g_P(r', t', C_P(h, t)))]
\end{split}
\end{equation}
where $\mathcal{K}' = \{h', r, t\}$ is the corrupted triples by replacing the head entity with an arbitrary entity, $n$ is the number of negative samples and $\sigma(\cdot)$ is the logistic function. $P(t|C_P(h,t),h;\Theta)$ in Eq.~\eqref{P_t} and $P(r|h,t;\Theta)$ in Eq.~\eqref{P_r} are approximated likewise.

In real data sets, the size of neighbor context and path context may be very large, which is computationally expensive for model learning. For this reason, we sample from neighbor context and path context to make triple context tractable. Specifically, we set a threshold $n_N$ for neighbor context and $n_P$ for path context; if the size of the original context exceeds the threshold, we sample a subset, size of which is the threshold, for model learning. Moreover, the length of relation path is constrained to 2 and 3 in our model.

\begin{equation}
\begin{split}
P(\mathcal{K}|\Theta) &= \prod_{(h,r,t) \in \mathcal{K}} f(h,r,t) \\
&= \prod_{(h, r, t) \in \mathcal{K}} {P(h|C(h,r,t);\Theta) \cdot P(r,t|C(h,r,t), h; \Theta)} \\
&\approx \prod_{(h, r, t) \in \mathcal{K}} [\sigma(g_N(h, C_N(h))) \cdot \prod_{h'} \sigma(-g_N(h', C_N(h)))] \\
&\cdot [\sigma(g_P(r, t, C_P(h, t))) \cdot \prod_{r', t'} \sigma(- g_P(r', t', C_P(h, t)))]
\end{split}
\end{equation}

\begin{comment}
the objective function is formulated as follows:
\begin{equation}\label{cost_function}
  \mathcal{L}(\mathcal{K}) = \sum_{(h,r,t)\in\mathcal{K}}
\end{equation}

For computational convenience, the original joint probability in Eq.~\eqref{joint_prob} is transformed into the negative logarithmic form, which can be optimized by stochastic gradient descent (SGD). The cost function of a triple $(h,r,t)$ is formulated as:
\begin{equation}\label{score_function}
*
\end{equation}
\end{comment}


\section{Experiments}\label{sec:set}

\subsection{Data Sets}






\input{rel}
\section{Conclusion}\label{sec:con}
In this paper, we proposed TCE, a KG embedding model which is able to take advantages of the triple context in the graph. By defining two kinds of context of a triple and representing them in a unified framework, our model can learn embeddings that are aware of their context. We evaluate our model on link prediction and the experimental results show significant improvements over the major baselines.

In the future, we will research on the following aspects: (1) Conduct experiments on more data sets and tasks to validate our model. (2) Current results show complementarity to some other methods such as TransH, TransR. We would think about a combination of those methods and our model.



%% The file named.bst is a bibliography style file for BibTeX 0.99c
\bibliographystyle{named}
%\bibliographystyle{unsrt}
\bibliography{bibOfTex}

\end{document}

