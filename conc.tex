\section{Conclusion}\label{sec:con}
In this paper, we proposed ATCE, a new KG embedding model which is able to take advantages of the triple context which includes neighbor context and path context in the KG. By defining two kinds of context of a triple and representing them in a unified framework, our model can learn embeddings that are aware of their context. ATCE not only can learn the embedding of a given KG, but also can capture the complex structure in KG. We evaluate our model on two benchmark datasets on link prediction and triple classification. In the meantime, we analysis the attention mechanism for choosing the neighbor enmities and connective pathes in triple context. The experimental results show significant improvements over the major baselines.

In the future, we will research on the following aspects: (1) it will be interesting to incorporate multilingual knowledge graph in to our model to further improve the performance. (2) Current results show complementarity to some other methods such as TransH, TransR. We would think about a combination of those methods and our model. 
